\subsection{Kiểm thử từng thành phần}

\begin{frame}[t,fragile]
\frametitle{Kiểm thử Module 4G/GPS \& Khởi tạo hệ thống}
\begin{columns}[T]
    %----------------- Column Log -----------------%
    \column{0.55\textwidth}
    \begin{minted}[fontsize=\scriptsize,breaklines]{text}
I (2329)  SIM_4G: Received: Quectel EG800K
OK
I (5329)  SIM_4G: Received: +CSQ: 31,99
OK
I (28339) SIM_4G: Received: +QIACT: 1,1,1,"9.204.251.200"
OK
I (34349) SIM_4G: Received: +QGPSLOC: 10.88862,106.77975
OK
I (9961)  SIM4G_AT: Initializing SIM4G AT driver...
I (10011) APP_MAIN: System initialization complete.
I (10021) APP_MAIN: Application started successfully
    \end{minted}

    %----------------- Column Phân tích -----------------%
    \column{0.45\textwidth}
    \begin{itemize}
        \item \textbf{Mục đích:} 
        \begin{itemize}
            \item Xác nhận giao tiếp AT command, kết nối 4G và thu nhận GPS.
            \item Kiểm tra các module SIM4G-GPS, Wi-Fi và các tác vụ chính.
        \end{itemize}
        \item \textbf{Quy trình:}
        \begin{enumerate}
            \item Khởi tạo UART, kiểm tra modem và SIM.
            \item Đánh giá chất lượng sóng (\texttt{+CSQ}).
            \item Cấu hình APN, kích hoạt PDP context.
            \item Bật GPS và truy vấn tọa độ.
        \end{enumerate}
        \item \textbf{Kết luận:} 
        \begin{itemize}
            \item Module 4G/GPS hoạt động ổn định, sẵn sàng gửi cảnh báo và thông tin định vị.
            \item Hệ thống khởi tạo thành công, tích hợp phần mềm và phần cứng ổn định.
        \end{itemize}
    \end{itemize}
\end{columns}
\end{frame}
% ----------------------------
% Slide: Kiểm thử MQTT
% ----------------------------
\begin{frame}[t,fragile]
\frametitle{Kiểm thử MQTT và truyền dữ liệu}
\begin{columns}[T]
    \column{0.55\textwidth}
    \begin{itemize}
        \item \textbf{Mục đích:} Kiểm tra kết nối broker MQTT và gửi bản tin định kỳ.
        \item \textbf{Log tiêu biểu:}
        \begin{minted}[fontsize=\footnotesize,breaklines]{text}
I (19961) USER_MQTT: MQTT_EVENT_CONNECTED
I (39991) JSON_WRAPPER: {"device_id":"ESP32_DEV_76E48B","fall_detected":false,...}
        \end{minted}
        \item \textbf{Kết luận:} Kết nối MQTT ổn định, dữ liệu được truyền tới hệ thống giám sát.
    \end{itemize}
    \column{0.45\textwidth}
    \begin{figure}
        \centering
        \includegraphics[width=0.75\linewidth]{json_data_dashboard.png}
        \caption{Dashboard hiển thị bản tin MQTT.}
    \end{figure}
\end{columns}
\end{frame}

% ----------------------------
% Slide: Phát hiện té ngã & cảnh báo
% ----------------------------
\begin{frame}[t,fragile]
\frametitle{Phát hiện Té ngã \& Xử lý Cảnh báo}
\begin{columns}[T]
    \column{0.55\textwidth}
    \begin{itemize}
        \item \textbf{Quy trình:}
        \begin{enumerate}
            \item Thuật toán ghi nhận chuỗi trạng thái \texttt{LOW\_G} $\to$ \texttt{HIGH\_G}.
            \item Kích hoạt cảnh báo: SMS, MQTT, buzzer, LED.
        \end{enumerate}
        \item \textbf{Log thực tế:}
        \begin{minted}[fontsize=\footnotesize,breaklines]{text}
\alert{E (159131) FALL_LOGIC: FALL DETECTED! Accel: 0.99 g}
I (159151) SIM4G_GPS: SMS request queued successfully
I (159881) SIM4G_AT: SMS sent successfully.
        \end{minted}
        \item \textbf{Kết luận:} Hệ thống phát hiện và xử lý cảnh báo thành công, đa kênh.
    \end{itemize}
    \column{0.45\textwidth}
    \begin{figure}
        \centering
        \includegraphics[width=0.75\linewidth]{module1_real_log.jpg}
        \caption{Log thực tế module cảm biến khi phát hiện té ngã.}
    \end{figure}
\end{columns}
\end{frame}

% ----------------------------
% Slide: Camera ESP32 – Luồng hình ảnh & nhận diện
% ----------------------------
\begin{frame}[t,fragile]
\frametitle{Camera ESP32-CAM: Luồng hình ảnh và xử lý}
\begin{columns}[T]
    %----------------- Column Hình -----------------%
    \column{0.55\textwidth}
    \begin{figure}[H]
        \centering
        \includegraphics[width=\linewidth]{module2_stream_example.jpg}
        \caption{Luồng hình ảnh từ ESP32-CAM qua HTTP.}
    \end{figure}

    %----------------- Column Phân tích -----------------%
    \column{0.45\textwidth}
    \begin{itemize}
        \item \textbf{Mục đích:} Kiểm tra kết nối Wi-Fi và phát luồng hình ảnh.
        \item \textbf{Xử lý Python:}
        \begin{itemize}
            \item Nhận luồng video từ ESP32-CAM.
            \item Tiền xử lý khung hình.
            \item TensorFlow Lite phát hiện người, vẽ skeleton.
        \end{itemize}
        \item \textbf{Kết quả:} Thời gian thực 3–5 FPS, hoạt động ổn định.
        \item \textbf{Ứng dụng:} Cảnh báo té ngã, đồng bộ MQTT, gửi thông báo Telegram.
    \end{itemize}
\end{columns}
\end{frame}

% ----------------------------
% Slide: Xử lý nhận diện hình ảnh (Python)
% ----------------------------
\begin{frame}[t]
\frametitle{Xử lý nhận diện hình ảnh (Python)}
\begin{itemize}
    \item \textbf{Mục đích:} Phát hiện người và trích xuất skeleton từ luồng video.
    \item \textbf{Quy trình:}
    \begin{itemize}
        \item Nhận luồng video từ ESP32-CAM.
        \item Tiền xử lý khung hình.
        \item TensorFlow Lite phát hiện người, vẽ skeleton.
    \end{itemize}
    \item \textbf{Kết quả:} Thời gian thực 3–5 FPS, hoạt động ổn định.
\end{itemize}
\end{frame}

% ----------------------------
% Slide: Kết quả xử lý hình ảnh
% ----------------------------
\begin{frame}[t]
\frametitle{Kết quả xử lý hình ảnh}
\begin{figure}
    \centering
    \includegraphics[width=0.75\linewidth]{fall_detection_screen_shoot.png}
    \caption{Module Python phát hiện người và vẽ skeleton.}
\end{figure}
\end{frame}

% ----------------------------
% Slide: Log thực nghiệm Python
% ----------------------------
\begin{frame}[fragile]
\frametitle{Log thực nghiệm Python}
\begin{figure}[H]
    \centering
    \includegraphics[width=0.75\linewidth]{python_runing_log.png}
    \caption{Log Python xử lý luồng camera và cảnh báo.}
\end{figure}
\begin{itemize}
    \item \textbf{Kết luận:} Module Python hoạt động ổn định, đồng bộ MQTT và kích hoạt cảnh báo thành công.
\end{itemize}
\end{frame}

% ----------------------------
% Slide: Đánh giá dữ liệu cảm biến MPU6050
% ----------------------------
\begin{frame}[fragile]
\frametitle{Đánh giá dữ liệu cảm biến (MPU6050)}
\begin{itemize}
    \item \textbf{Quy trình:} Giả lập té ngã, so sánh dữ liệu bình thường và té ngã.
    \item \textbf{Kết quả:}
    \begin{itemize}
        \item \textbf{Gyro (dps):} Bình thường $\approx \pm 1.5$; té ngã tăng $\approx \pm 250$.
        \item \textbf{Accel (g):} Bình thường $\approx 0.93$; té ngã $-2.0$ đến $+1.0$.
    \end{itemize}
    \item \textbf{Kết luận:} \textbf{Accel\_Mag} và \textbf{Gyro\_Mag} là chỉ báo hiệu quả.
\end{itemize}
\end{frame}

% ----------------------------
% Slide: Biến thiên Gyro_Mag
% ----------------------------

\begin{frame}[t,fragile]
\frametitle{Biến thiên & Phân tích Gyro\_Mag}
\begin{columns}[T]
    \column{0.55\textwidth}
    \begin{figure}[H]
        \centering
        \includegraphics[width=\linewidth]{gyro_time.png}
        \caption{Magnitude gia tốc góc theo thời gian. Peak biểu thị té ngã.}
    \end{figure}
    \column{0.45\textwidth}
    \begin{itemize}
        \item \textbf{Bình thường:} Dao động nhỏ ($\lesssim 2$ dps).
        \item \textbf{Sự kiện té ngã:} Xuất hiện đỉnh lớn (hàng chục–trăm dps).
        \item \textbf{Hậu té:} Giảm nhanh về mức nền, trạng thái nằm yên.
    \end{itemize}
\end{columns}
\end{frame}
% ----------------------------
% Slide: Biến thiên Accel_Mag

\begin{frame}[t,fragile]
\frametitle{Biến thiên & Phân tích Accel\_Mag}
\begin{columns}[T]
    %----------------- Column Hình -----------------%
    \column{0.55\textwidth}
    \begin{figure}[H]
        \centering
        \includegraphics[width=\linewidth]{accel_time.png}
        \caption{Magnitude gia tốc (Accel\_Mag) theo thời gian. Xung lớn biểu thị té ngã.}
    \end{figure}
    %----------------- Column Phân tích -----------------%
    \column{0.45\textwidth}
    \begin{itemize}
        \item \textbf{Bình thường:} Gần 1\,g, dao động nhỏ.
        \item \textbf{Sự kiện té ngã:} Xuất hiện xung hoặc thay đổi đột ngột, vượt hoặc giảm mạnh so với 1\,g.
        \item \textbf{Hậu té:} Trở về gần 1\,g nhưng phân bố vector khác (tư thế nằm).
    \end{itemize}
\end{columns}
\end{frame}
% ----------------------------
% Slide: Kiểm thử cảnh báo Asterisk AMI
% ----------------------------
\begin{frame}[fragile]
\frametitle{Kiểm thử cảnh báo qua Asterisk AMI}
\begin{itemize}
    \item \textbf{Mục đích:} Kiểm chứng cuộc gọi và SMS tự động.
    \item \textbf{Kết quả:}
    \begin{itemize}
        \item Kết nối thành công tới Asterisk AMI.
        \item SMS tới \texttt{6001, 6002} thành công; \texttt{6003} lỗi.
        \item Lỗi gọi điện không ảnh hưởng đến cảnh báo chính.
    \end{itemize}
\end{itemize}
\begin{figure}[H]
    \centering
    \includegraphics[width=0.75\linewidth]{ast_call_sms_test.png}
    \caption{Thử nghiệm chức năng gọi và nhắn tin.}
\end{figure}
\end{frame}

% ----------------------------
% Slide: Kiểm thử cảnh báo Telegram
% ----------------------------
\begin{frame}[t,fragile]
\frametitle{Kiểm thử cảnh báo qua Telegram}
\begin{itemize}
    \item \textbf{Cơ chế hoạt động:}
    \begin{itemize}
        \item \textbf{Phần cứng:} MQTT từ thiết bị -> trung tâm -> Telegram.
        \item \textbf{Python:} Camera -> Python gửi cảnh báo trực tiếp Telegram kèm hình.
    \end{itemize}
\end{itemize}
\end{frame}

% ----------------------------
% Slide: Thông báo Telegram
% ----------------------------
\begin{frame}[t,fragile]
\frametitle{Thông báo Telegram}
\begin{columns}[T]
    \column{0.5\textwidth}
    \begin{figure}[H]
        \centering
        \includegraphics[width=0.75\linewidth]{telegram_fall_module1_send.png}
        \caption{Thông báo từ phần cứng.}
    \end{figure}
    \column{0.5\textwidth}
    \begin{figure}[H]
        \centering
        \includegraphics[width=0.75\linewidth]{telegram_python_fall_send.png}
        \caption{Thông báo từ Python.}
    \end{figure}
    \vspace{2mm}
    \begin{itemize}
        \item \textbf{Kết luận:} Kênh Telegram hoạt động ổn định, đa nguồn cảnh báo tăng độ tin cậy.
    \end{itemize}
\end{columns}
\end{frame}
