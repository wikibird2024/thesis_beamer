
% ----------------------------
% Slide 1: Kết luận chi tiết & hướng phát triển
% ----------------------------
\begin{frame}{Kết quả chung \& Hướng phát triển}
\begin{columns}[T]
    % Column 1: Kết quả
    \column{0.5\textwidth}
    \textbf{Kết quả chính}
    \begin{itemize}
        \item Hệ thống IoT: ESP32, MPU6050, SIM4G--GPS, buzzer, LED, Python/FreeRTOS.
        \item Thuật toán phát hiện té ngã: sensor fusion dữ liệu cảm biến + ảnh, thời gian thực.
        \item Cơ chế cảnh báo đa dạng: buzzer, LED, SMS, MQTT--Telegram, SIP.
        \item Thực nghiệm: độ chính xác tương đối, độ trễ thấp, hoạt động ổn định.
        \item Tiềm năng ứng dụng: giám sát sức khỏe từ xa
    \end{itemize}

    % Column 2: Hạn chế & hướng phát triển
    \column{0.5\textwidth}
    \textbf{Hạn chế \& hướng phát triển}
    \begin{itemize}
        \item Quy mô thử nghiệm hạn chế; số lượng thiết bị & kênh truyền còn nhỏ.
        \item Thuật toán chưa hoàn toàn ổn định; thiết bị còn cồng kềnh.
        \item Kết nối 4G/GPS chưa tối ưu và tin cậy
        \item \textbf{Hướng phát triển}: mở rộng thử nghiệm, cải tiến thuật toán, thiết bị nhỏ gọn, hoàn thiện kết nối 4G/GPS, tích hợp nền tảng IoT.
    \end{itemize}
\end{columns}
\end{frame}

% ----------------------------
% Slide 2: Kết luận chung + cảm ơn
% ----------------------------
\begin{frame}{Kết luận chung \& Cảm ơn}
\begin{itemize}
    \item FDAS nguyên mẫu hoạt động thành công: phát hiện té ngã & gửi cảnh báo thời gian thực.
    \item Độ trễ đầu-cuối < 5 giây; hệ thống ổn định, hiệu năng đáp ứng yêu cầu.
    \item Nghiên cứu mang lại kiến thức & kinh nghiệm thực tiễn về IoT, nhúng, sensor fusion, hệ thống phân tán.
    \item Xác định các thách thức thực tế: tối ưu phần cứng, nâng cao độ chính xác thuật toán, ổn định truyền thông.
\end{itemize}

\vspace{0.5cm}
\centering
\textbf{Cảm ơn quý thầy/cô và các bạn đã chú ý theo dõi}
\end{frame}
