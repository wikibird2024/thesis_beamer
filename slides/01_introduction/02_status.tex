
% --- Bắt đầu SUBSECTION: Nghiên cứu liên quan ---
\subsection{Tình hình nghiên cứu}
% --- Slide 4: Tổng quan các phương pháp ---
\begin{frame}{Các phương pháp phát hiện té ngã}
    \begin{itemize}
        \item \textbf{Dựa trên thị giác (Vision-based)}: Sử dụng camera và thuật toán nhận diện tư thế người (MediaPipe, OpenPose).
        \item \textbf{Dựa trên cảm biến đeo (Wearable-based)}: Dùng cảm biến quán tính (IMU, MPU6050) trên thiết bị.
        \item \textbf{Kết hợp đa phương thức (Multi-modal)}: Tích hợp dữ liệu từ nhiều nguồn để tăng độ chính xác.
    \end{itemize}
\end{frame}


% --- Slide 5: Nghiên cứu trong và ngoài nước ---
\begin{frame}{Nghiên cứu trong và ngoài nước}
    \begin{columns}[T]
        \begin{column}{0.48\textwidth}
            \textbf{Quốc tế}
            \begin{itemize}
                \item \textbf{Xu hướng}: Sử dụng YOLO, Transformer, AI nhẹ, cảm biến mmWave.
                \item \textbf{Thành tựu}: Giảm false alarm, tối ưu cho thiết bị biên, Sensor Fusion.
            \end{itemize}
        \end{column}
        \begin{column}{0.48\textwidth}
            \textbf{Trong nước}
            \begin{itemize}
                \item \textbf{Thực trạng}: Chủ yếu mô hình thử nghiệm (PoC) với ESP32, Arduino.
                \item \textbf{Hạn chế}: Thiếu dữ liệu lớn, độ chính xác thấp (75-85\%), thiếu tích hợp đa phương thức.
            \end{itemize}
        \end{column}
    \end{columns}
\end{frame}

