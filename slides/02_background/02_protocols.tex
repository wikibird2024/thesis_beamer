
% Slide 1: Tổng quan
\begin{frame}
\frametitle{Các Giao Thức Truyền Thông trong Hệ Thống Cảnh Báo IoT}
\begin{center}
\Large Hệ thống phát hiện ngã với ba giao thức cốt lõi
\end{center}

\begin{itemize}
\item \textbf{SIP}: Truyền tải âm thanh/video cảnh báo thời gian thực
\item \textbf{MQTT}: Vận chuyển dữ liệu cảm biến từ thiết bị IoT  
\item \textbf{JSON}: Định dạng cấu trúc dữ liệu trao đổi
\end{itemize}

\begin{block}{Mục tiêu}
Xây dựng hệ thống cảnh báo không gián đoạn, độ trễ thấp từ cảm biến đến cuộc gọi VoIP
\end{block}
\end{frame}

% SIP
\begin{frame}
\frametitle{Giao thức SIP - Khởi tạo Phiên}
\textbf{Chức năng chính:}
\begin{itemize}
\item Thiết lập cuộc gọi VoIP từ hệ thống cảnh báo
\item Kết nối với Asterisk PBX để gọi điện thoại
\item Truyền âm thanh cảnh báo qua RTP
\end{itemize}
\textbf{Các bước hoạt động:}
\begin{enumerate}
\item REGISTER: Đăng ký thiết bị với server
\item INVITE: Khởi tạo cuộc gọi cảnh báo
\item ACK: Xác nhận kết nối thành công
\item RTP: Truyền dữ liệu âm thanh
\item BYE: Kết thúc cuộc gọi
\end{enumerate}
\begin{alertblock}{Lưu ý}
Sử dụng ICE để xuyên NAT, TLS/SRTP để bảo mật
\end{alertblock}
\end{frame}

\begin{frame}
\frametitle{Phân biệt Đường tín hiệu và Đường truyền phương tiện}

\begin{block}{Đường tín hiệu (Signaling Path)}
\begin{itemize}
\item Mang các tin nhắn SIP (INVITE, BYE, 200 OK, v.v.)
\item Thiết lập, quản lý và kết thúc cuộc gọi
\item Sử dụng TCP hoặc UDP
\end{itemize}
\end{block}

\begin{block}{Đường truyền phương tiện (Media Path)}
\begin{itemize}
\item Mang dữ liệu thoại/video thực tế
\item Sử dụng RTP qua UDP
\item Truyền trực tiếp giữa các điểm cuối
\end{itemize}
\end{block}

\end{frame}

\begin{frame}
\frametitle{Sơ đồ Đường tín hiệu và Đường truyền phương tiện}

\begin{figure}[h]
\centering
\begin{tikzpicture}[
    box/.style={rectangle, draw, rounded corners, minimum height=1em, minimum width=2.5em, align=center, fill=blue!10}, 
    signal_arrow/.style={-Stealth, thick, blue!70!white, shorten >=1pt, shorten <=1pt}, 
    media_arrow/.style={-Stealth, thick, red!70!white, shorten >=1pt, shorten <=1pt}, 
    label_style/.style={font=\tiny, align=center, text=black},
    node distance=1.8cm and 2.5cm 
]
    % Các Node
    \node[box] (A) {SIP Endpoint A};
    \node[box, right=of A] (Proxy) {SIP Proxy/PBX}; 
    \node[box, right=of Proxy] (B) {SIP Endpoint B};
    \node[box, below=1.5cm of Proxy] (MediaProxy) {Media Proxy}; 
    
    % Đường tín hiệu
    \draw[signal_arrow] (A) -- node[above=1pt, label_style] {SIP (TCP/UDP)} (Proxy);
    \draw[signal_arrow] (Proxy) -- node[above=1pt, label_style] {SIP (TCP/UDP)} (B);
    
    % Đường truyền phương tiện
    \draw[media_arrow] (A) -- node[midway, left=1pt, label_style] {RTP (UDP)} (MediaProxy); 
    \draw[media_arrow] (MediaProxy) -- node[midway, right=1pt, label_style] {RTP (UDP)} (B); 
\end{tikzpicture}
\end{figure}

\end{frame}

\begin{frame}
\frametitle{Giao thức ICE (Interactive Connectivity Establishment)}

\begin{block}{Vấn đề}
Các thiết bị thường nằm sau NAT/firewall, ngăn cản truyền dữ liệu RTP trực tiếp
\end{block}

\begin{block}{Giải pháp ICE}
\begin{itemize}
\item \textbf{Local IP:} Địa chỉ IP nội bộ của thiết bị
\item \textbf{STUN:} Phát hiện địa chỉ IP công cộng và cổng NAT
\item \textbf{TURN:} Máy chủ chuyển tiếp khi STUN thất bại
\end{itemize}
\end{block}

\begin{alertblock}{Lưu ý}
Quá trình ICE thực hiện qua SDP trong thông điệp SIP
\end{alertblock}

\end{frame}

\begin{frame}
\frametitle{SIP trong Hệ Thống Cảnh Báo Asterisk}

\begin{block}{Lợi ích}
\begin{itemize}
\item \textbf{Quản lý tập trung:} Đồng nhất cấu hình và quản lý thiết bị
\item \textbf{Tương thích cao:} Hỗ trợ đa dạng nền tảng và thiết bị
\item \textbf{Chuẩn mở:} Tích hợp dễ dàng với hạ tầng hiện có
\item \textbf{Bảo mật:} Hỗ trợ TLS (SIP) và SRTP (RTP)
\end{itemize}
\end{block}

\begin{block}{Vai trò của Asterisk}
Đóng vai trò như SIP server, xử lý đăng ký và định tuyến cuộc gọi
\end{block}

\end{frame}

\begin{frame}
\frametitle{Luồng Phân Phối Cảnh Báo (1/2)}

\begin{block}{1. Đăng ký}
\begin{itemize}
\item Thiết bị SIP gửi REGISTER tới Asterisk
\item Xác thực qua username/password trong header Authorization
\end{itemize}
\end{block}

\begin{block}{2. Thiết lập phiên}
\begin{itemize}
\item Ứng dụng Linux ra lệnh \texttt{Originate} qua AMI
\item Asterisk gửi INVITE → 180 Ringing → 200 OK → ACK
\item Sử dụng header \texttt{Alert-Info: ;info=alert-autoanswer}
\end{itemize}
\end{block}

\end{frame}

\begin{frame}
\frametitle{Luồng Phân Phối Cảnh Báo (2/2)}

\begin{block}{3. Trao đổi dữ liệu}
\begin{itemize}
\item Dữ liệu cảnh báo truyền qua RTP
\item TTS (Text-to-Speech) tạo âm thanh từ văn bản
\item Sử dụng codec G.711 hoặc G.729
\end{itemize}
\end{block}

\begin{block}{4. Kết thúc phiên}
\begin{itemize}
\item Một bên gửi BYE
\item Bên kia phản hồi 200 OK
\end{itemize}
\end{block}

\end{frame}

\begin{frame}
\frametitle{Tích Hợp SMS qua SIP}

\begin{block}{Cấu hình}
\begin{itemize}
\item Kích hoạt \texttt{textsupport=yes} trong \texttt{sip.conf}
\item Định nghĩa logic xử lý trong \texttt{extensions.conf}
\end{itemize}
\end{block}

\begin{block}{Thực hiện}
\begin{itemize}
\item Sử dụng lệnh \texttt{MessageSend}
\item Gửi tin nhắn SIP MESSAGE
\item Tích hợp SMS gateway để chuyển tiếp sang mạng di động
\end{itemize}
\end{block}

\end{frame}

% Slide 3: Giao thức MQTT
\begin{frame}
\frametitle{Giao thức MQTT - Tổng quan}

\begin{block}{Định nghĩa}
\begin{itemize}
\item MQTT = Message Queuing Telemetry Transport
\item Giao thức nhẹ, tối ưu cho IoT và M2M
\item Hoạt động trên TCP/IP với cơ chế kết nối lâu dài
\end{itemize}
\end{block}

\begin{block}{Đặc điểm}
\begin{itemize}
\item Thiết kế cho thiết bị có tài nguyên hạn chế
\item Phù hợp với băng thông thấp
\item Hỗ trợ kết nối không ổn định
\item Tiêu chuẩn OASIS cho IoT messaging
\end{itemize}
\end{block}

\end{frame}

\begin{frame}
\frametitle{Kiến trúc Publish/Subscribe của MQTT}

\begin{figure}[h]
\centering
\begin{tikzpicture}[
    broker/.style={rectangle, draw, rounded corners, minimum height=2em, minimum width=3em, align=center, fill=green!20}, 
    client/.style={rectangle, draw, rounded corners, minimum height=1.5em, minimum width=2.5em, align=center, fill=blue!10}, 
    pub_arrow/.style={-Stealth, thick, orange!70!white, shorten >=1pt, shorten <=1pt}, 
    sub_arrow/.style={-Stealth, thick, purple!70!white, shorten >=1pt, shorten <=1pt}, 
    label_style/.style={font=\tiny, align=center, text=black},
    node distance=2cm
]
    % Broker trung tâm
    \node[broker] (broker) {MQTT Broker \\ (Mosquitto/HiveMQ)};
    
    % Publishers
    \node[client, above left=of broker] (pub1) {Publisher 1};
    \node[client, left=of broker] (pub2) {Publisher 2};
    
    % Subscribers
    \node[client, above right=of broker] (sub1) {Subscriber 1};
    \node[client, right=of broker] (sub2) {Subscriber 2};
    \node[client, below right=of broker] (sub3) {Subscriber 3};
    
    % Publish arrows
    \draw[pub_arrow] (pub1) -- node[above left, label_style] {PUBLISH \\ topic/data} (broker);
    \draw[pub_arrow] (pub2) -- node[left, label_style] {PUBLISH} (broker);
    
    % Subscribe arrows
    \draw[sub_arrow] (broker) -- node[above right, label_style] {SUBSCRIBE \\ topic} (sub1);
    \draw[sub_arrow] (broker) -- node[right, label_style] {MESSAGE} (sub2);
    \draw[sub_arrow] (broker) -- node[below right, label_style] {MESSAGE} (sub3);
\end{tikzpicture}
\end{figure}

\end{frame}

\begin{frame}
\frametitle{Lợi ích của mô hình Publish/Subscribe}

\begin{block}{Tách rời không gian (Space Decoupling)}
\begin{itemize}
\item Publisher và Subscriber không cần biết địa chỉ IP của nhau
\item Giao tiếp thông qua broker trung tâm
\end{itemize}
\end{block}

\begin{block}{Tách rời thời gian (Time Decoupling)}
\begin{itemize}
\item Không yêu cầu kết nối đồng thời
\item Hỗ trợ \textbf{retained messages} cho subscriber mới
\item Quản lý \textbf{clean session} cho trạng thái client
\end{itemize}
\end{block}

\begin{block}{Tách rời đồng bộ (Synchronization Decoupling)}
\begin{itemize}
\item Truyền và nhận hoạt động độc lập
\item Giảm độ trễ và tăng hiệu suất
\end{itemize}
\end{block}

\end{frame}


\begin{frame}
\frametitle{Quality of Service (QoS)}

\begin{block}{QoS 0 - At most once}
\begin{itemize}
\item "Fire and forget" - không có xác nhận
\item Nhanh nhất nhưng có thể mất message
\item Phù hợp cho dữ liệu cảm biến thường xuyên
\end{itemize}
\end{block}

\begin{block}{QoS 1 - At least once}
\begin{itemize}
\item Đảm bảo message được gửi ít nhất một lần
\item Có thể trùng lặp message
\item Cân bằng giữa độ tin cậy và hiệu suất
\end{itemize}
\end{block}

\begin{block}{QoS 2 - Exactly once}
\begin{itemize}
\item Đảm bảo message được gửi đúng một lần
\item Chậm nhất nhưng tin cậy nhất
\item Sử dụng cho dữ liệu quan trọng
\end{itemize}
\end{block}

\end{frame}


\begin{frame}
\frametitle{Bảo mật trong MQTT}

\begin{block}{Mã hóa Transport Layer}
\begin{itemize}
\item Hỗ trợ TLS/SSL cho kết nối bảo mật
\item MQTT over TLS (port 8883)
\item Bảo vệ dữ liệu trong quá trình truyền
\end{itemize}
\end{block}

\begin{block}{Xác thực và Ủy quyền}
\begin{itemize}
\item Username/Password authentication
\item Client certificates cho xác thực mạnh
\item Access Control Lists (ACL) kiểm soát quyền truy cập topic
\end{itemize}
\end{block}
\end{frame}

\begin{frame}
\frametitle{Các lệnh MQTT chính}

\begin{block}{Connection Management}
\begin{itemize}
\item \texttt{CONNECT}: Khởi tạo kết nối với broker
\item \texttt{CONNACK}: Xác nhận kết nối từ broker
\item \texttt{DISCONNECT}: Ngắt kết nối một cách graceful
\end{itemize}
\end{block}

\begin{block}{Messaging Operations}
\begin{itemize}
\item \texttt{PUBLISH}: Gửi message tới topic
\item \texttt{PUBACK/PUBREC/PUBREL/PUBCOMP}: QoS acknowledgments
\item \texttt{SUBSCRIBE}: Đăng ký nhận messages từ topic
\item \texttt{SUBACK}: Xác nhận subscription
\item \texttt{UNSUBSCRIBE}: Hủy đăng ký topic
\end{itemize}
\end{block}

\begin{block}{Keep Alive}
\begin{itemize}
\item \texttt{PINGREQ/PINGRESP}: Duy trì kết nối
\end{itemize}
\end{block}

\end{frame}

\begin{frame}
\frametitle{MQTT trong Hệ thống IoT và Cảnh báo}

\begin{block}{Ứng dụng trong IoT}
\begin{itemize}
\item Thu thập dữ liệu từ sensors
\item Điều khiển thiết bị từ xa
\item Giám sát trạng thái hệ thống
\item Gửi thông báo và cảnh báo
\end{itemize}
\end{block}

\begin{block}{Lợi ích cho hệ thống cảnh báo}
\begin{itemize}
\item Kết nối đáng tin cậy với thiết bị IoT
\item Truyền dữ liệu real-time
\item Hỗ trợ offline messaging
\item Scale tốt với nhiều thiết bị
\end{itemize}
\end{block}

\end{frame}

\begin{frame}
\frametitle{Giao thức MQTT - Truyền Dữ Liệu Cảm Biến}

\textbf{Đặc điểm:}
\begin{itemize}
\item Nhẹ, tiết kiệm băng thông cho thiết bị IoT
\item Mô hình Publish/Subscribe qua Broker
\item Hỗ trợ 3 mức QoS đảm bảo độ tin cậy
\end{itemize}

\textbf{Mức QoS:}
\begin{itemize}
\item \textcolor{green}{\textbf{QoS 0}}: Gửi một lần (dữ liệu thường)
\item \textcolor{orange}{\textbf{QoS 1}}: Ít nhất một lần (có xác nhận)
\item \textcolor{red}{\textbf{QoS 2}}: Đúng một lần (cảnh báo quan trọng)
\end{itemize}

\textbf{Ví dụ topic:} \texttt{sensor/room/temperature}, \texttt{alert/fall/detected}
\end{frame}

%----------------------JSON--------------------%
% slides/02_background/07_background_json.tex

\begin{frame}
\frametitle{JSON - JavaScript Object Notation}
\begin{block}{Định nghĩa}
\begin{itemize}
\item JSON: JavaScript Object Notation.
\item Định dạng dữ liệu nhẹ, dễ đọc, dùng để trao đổi dữ liệu.
\item Độc lập với ngôn ngữ, dựa trên cú pháp JavaScript.
\end{itemize}
\end{block}

\begin{block}{Đặc điểm}
\begin{itemize}
\item Định dạng dễ đọc, dùng cho lưu trữ và truyền dữ liệu.
\item Cấu trúc gồm cặp \texttt{key:value}, dùng \texttt{\{\}}.
\item Chuẩn phổ biến trong ứng dụng IoT.
\end{itemize}
\end{block}

\begin{block}{Tiêu chuẩn}
RFC 8259: Chuẩn Internet cho định dạng trao đổi dữ liệu JSON.
\end{block}
\end{frame}

\begin{frame}[fragile]
\frametitle{Cấu trúc JSON cơ bản}
\begin{block}{Cấu trúc dữ liệu}
\begin{itemize}
\item \textbf{Object}: \texttt{\{key: value\}}
\item \textbf{Array}: \texttt{[value1, value2]}
\item \textbf{Kiểu giá trị}: String, Number, Boolean, null, Object, Array.
\end{itemize}
\end{block}

\begin{exampleblock}{Ví dụ JSON}
\begin{minted}[frame=single, linenos, breaklines, fontsize=\footnotesize]{json}
{
  "device_id": "ESP32_001",
  "temperature": 25.5,
  "sensors": ["temp", "light"]
}
\end{minted}
\end{exampleblock}
\end{frame}

\begin{frame}
\frametitle{Ứng dụng JSON trong IoT}
\begin{block}{Lưu cấu hình}
\begin{itemize}
\item Cấu hình thiết bị IoT (Wi-Fi, MQTT).
\item Lưu thông số cảm biến và máy chủ.
\end{itemize}
\end{block}

\begin{block}{Trao đổi dữ liệu}
\begin{itemize}
\item Định dạng payload cho MQTT, API.
\item Gửi thông báo cảnh báo trong hệ thống.
\end{itemize}
\end{block}

\begin{block}{Lợi ích}
\begin{itemize}
\item Tự mô tả, nhẹ hơn XML.
\item Tương thích đa nền tảng.
\end{itemize}
\end{block}
\end{frame}

\begin{frame}[fragile]
\frametitle{Ví dụ: Cấu hình ESP32}
\begin{minted}[frame=single, linenos, breaklines, fontsize=\footnotesize]{json}
{
  "network": {
    "ssid": "IoT_Network",
    "mqtt_broker": "192.168.1.100"
  },
  "device": {
    "id": "ESP32_001",
    "update_interval": 30
  }
}
\end{minted}
\end{frame}

\begin{frame}
\frametitle{Thư viện JSON cho hệ thống nhúng}

\begin{block}{json-c}
\begin{itemize}
\item Thư viện JSON cho C/C++.
\item Hỗ trợ phân tích cú pháp, tạo JSON.
\end{itemize}
\end{block}

\begin{block}{FirebaseJson}
\begin{itemize}
\item Thư viện dễ dùng, hỗ trợ JSON phức tạp.
\item Dựa trên cJSON, phù hợp IoT.
\end{itemize}
\end{block}
\end{frame}



\begin{frame}
\frametitle{JSON trong MQTT và SIP}
\begin{block}{JSON với MQTT}
\begin{itemize}
\item Payload JSON trong topic \texttt{sensor/data}.
\item Lưu cấu hình trong retained messages.
\end{itemize}
\end{block}

\begin{block}{JSON với SIP}
\begin{itemize}
\item Dữ liệu JSON trong custom headers, SIP MESSAGE.
\item Lưu cấu hình ứng dụng SIP.
\end{itemize}
\end{block}

\begin{alertblock}{Lưu ý}
JSON đảm bảo tương tác giữa các giao thức trong hệ sinh thái IoT.
\end{alertblock}
\end{frame}


% Slide 4: JSON và Tích hợp Hệ thống
\begin{frame}
\frametitle{JSON và Tích Hợp Hệ Thống}

\textbf{JSON - Định dạng dữ liệu:}
\begin{itemize}
\item Nhẹ, dễ đọc, tương thích đa nền tảng
\item Cấu hình thiết bị và trao đổi dữ liệu cảm biến
\item Tối ưu payload cho MQTT
\end{itemize}

\textbf{Luồng tích hợp hoàn chỉnh:}
\begin{enumerate}
\item ESP32 phát hiện ngã → tạo JSON payload
\item Gửi qua MQTT topic với QoS phù hợp  
\item Ứng dụng trung gian nhận và xử lý JSON
\item Kích hoạt cuộc gọi SIP qua Asterisk AMI
\item Phát cảnh báo âm thanh đến điện thoại
\end{enumerate}
\end{frame}

% Slide 5: Kết luận và Tối ưu
\begin{frame}
\frametitle{Kết Luận và Tối Ưu Hóa kết hợp cá phương thức}

\textbf{Lợi ích của việc kết hợp 3 giao thức:}
\begin{itemize}
\item \textbf{MQTT}: Thu thập dữ liệu hiệu quả từ cảm biến
\item \textbf{JSON}: Cấu trúc dữ liệu linh hoạt, dễ xử lý
\item \textbf{SIP}: Cảnh báo âm thanh tức thì, đáng tin cậy
\end{itemize}

\textbf{Các biện pháp tối ưu:}
\begin{itemize}
\item Payload JSON nhỏ gọn tiết kiệm năng lượng
\item QoS MQTT phù hợp với mức độ quan trọng
\item Tự động kết nối lại khi mất kết nối
\item Bảo mật TLS cho MQTT và SIP
\end{itemize}

\begin{block}{Kết quả}
Hệ thống cảnh báo tự động, tin cậy từ thiết bị nhúng đến cuộc gọi VoIP
\end{block}
\end{frame}

