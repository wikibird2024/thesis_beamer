
% Slide 1: Tổng quan
\begin{frame}
\frametitle{Các Giao Thức Truyền Thông trong Hệ Thống Cảnh Báo IoT}
\begin{center}
\Large Hệ thống phát hiện ngã với ba giao thức cốt lõi
\end{center}

\begin{itemize}
\item \textbf{SIP}: Truyền tải âm thanh/video cảnh báo thời gian thực
\item \textbf{MQTT}: Vận chuyển dữ liệu cảm biến từ thiết bị IoT  
\item \textbf{JSON}: Định dạng cấu trúc dữ liệu trao đổi
\end{itemize}

\begin{block}{Mục tiêu}
Xây dựng hệ thống cảnh báo không gián đoạn, độ trễ thấp từ cảm biến đến cuộc gọi VoIP
\end{block}
\end{frame}

% Slide 2: Giao thức SIP
\begin{frame}
\frametitle{Giao thức SIP - Khởi tạo Phiên}

\textbf{Chức năng chính:}
\begin{itemize}
\item Thiết lập cuộc gọi VoIP từ hệ thống cảnh báo
\item Kết nối với Asterisk PBX để gọi điện thoại
\item Truyền âm thanh cảnh báo qua RTP
\end{itemize}

\textbf{Các bước hoạt động:}
\begin{enumerate}
\item REGISTER: Đăng ký thiết bị với server
\item INVITE: Khởi tạo cuộc gọi cảnh báo
\item ACK: Xác nhận kết nối thành công
\item RTP: Truyền dữ liệu âm thanh
\item BYE: Kết thúc cuộc gọi
\end{enumerate}

\begin{alertblock}{Lưu ý}
Sử dụng ICE để xuyên NAT, TLS/SRTP để bảo mật
\end{alertblock}
\end{frame}

% Slide 3: Giao thức MQTT
\begin{frame}
\frametitle{Giao thức MQTT - Truyền Dữ Liệu Cảm Biến}

\textbf{Đặc điểm:}
\begin{itemize}
\item Nhẹ, tiết kiệm băng thông cho thiết bị IoT
\item Mô hình Publish/Subscribe qua Broker
\item Hỗ trợ 3 mức QoS đảm bảo độ tin cậy
\end{itemize}

\textbf{Mức QoS:}
\begin{itemize}
\item \textcolor{green}{\textbf{QoS 0}}: Gửi một lần (dữ liệu thường)
\item \textcolor{orange}{\textbf{QoS 1}}: Ít nhất một lần (có xác nhận)
\item \textcolor{red}{\textbf{QoS 2}}: Đúng một lần (cảnh báo quan trọng)
\end{itemize}

\textbf{Ví dụ topic:} \texttt{sensor/room/temperature}, \texttt{alert/fall/detected}
\end{frame}

% Slide 4: JSON và Tích hợp Hệ thống
\begin{frame}
\frametitle{JSON và Tích Hợp Hệ Thống}

\textbf{JSON - Định dạng dữ liệu:}
\begin{itemize}
\item Nhẹ, dễ đọc, tương thích đa nền tảng
\item Cấu hình thiết bị và trao đổi dữ liệu cảm biến
\item Tối ưu payload cho MQTT
\end{itemize}

\textbf{Luồng tích hợp hoàn chỉnh:}
\begin{enumerate}
\item ESP32 phát hiện ngã → tạo JSON payload
\item Gửi qua MQTT topic với QoS phù hợp  
\item Ứng dụng trung gian nhận và xử lý JSON
\item Kích hoạt cuộc gọi SIP qua Asterisk AMI
\item Phát cảnh báo âm thanh đến điện thoại
\end{enumerate}
\end{frame}

% Slide 5: Kết luận và Tối ưu
\begin{frame}
\frametitle{Kết Luận và Tối Ưu Hóa}

\textbf{Lợi ích của việc kết hợp 3 giao thức:}
\begin{itemize}
\item \textbf{MQTT}: Thu thập dữ liệu hiệu quả từ cảm biến
\item \textbf{JSON}: Cấu trúc dữ liệu linh hoạt, dễ xử lý
\item \textbf{SIP}: Cảnh báo âm thanh tức thì, đáng tin cậy
\end{itemize}

\textbf{Các biện pháp tối ưu:}
\begin{itemize}
\item Payload JSON nhỏ gọn tiết kiệm năng lượng
\item QoS MQTT phù hợp với mức độ quan trọng
\item Tự động kết nối lại khi mất kết nối
\item Bảo mật TLS cho MQTT và SIP
\end{itemize}

\begin{block}{Kết quả}
Hệ thống cảnh báo tự động, tin cậy từ thiết bị nhúng đến cuộc gọi VoIP
\end{block}
\end{frame}
