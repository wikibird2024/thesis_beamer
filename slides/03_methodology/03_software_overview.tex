\subsection{Triển khai thực hiện phần mềm cho hệ thống}
\begin{frame}
\frametitle{Kiến trúc Phần mềm Hệ thống Phát hiện Té ngã}

\begin{columns}[t]
\begin{column}{0.5\textwidth}
\begin{block}{Nguyên tắc Thiết kế}
Triển khai theo \textbf{kiến trúc mô-đun}:
\begin{itemize}
\item Tính linh hoạt cao
\item Dễ bảo trì và nâng cấp
\item Tách biệt chức năng
\end{itemize}
\end{block}

\begin{alertblock}{Thành phần chính}
\begin{itemize}
    \item Mã nhúng thiết bị
    \item Xử lý trung tâm server  
    \item Hệ thống cảnh báo
\end{itemize}
\end{alertblock}
\end{column}

\begin{column}{0.5\textwidth}
\begin{exampleblock}{Chi tiết Module}
\textbf{Server trung tâm:}
\begin{itemize}
\item Công nghệ: Python
\item Xử lý logic chung toàn hệ thống
\item Xử lý ảnh từ camera
\end{itemize}

\textbf{Module 1 - Cảm biến đeo:}
\begin{itemize}
\item Công nghệ: ESP-IDF + C
\item Thu thập dữ liệu chuyển động
\end{itemize}

\textbf{Module 2 - Camera:}
\begin{itemize}
\item Công nghệ: ESP-IDF + C  
\item Ghi hình và phát hiện bất thường
\end{itemize}
\end{exampleblock}
\end{column}
\end{columns}

\vspace{0.3cm}
\begin{center}
\textit{\small Mỗi thành phần thực hiện chức năng cụ thể và giao tiếp với nhau\\tạo thành hệ thống phát hiện té ngã hoàn chỉnh}
\end{center}

\end{frame}
