% Slide 1: Tổng quan Asterisk
\begin{frame}
\frametitle{Asterisk: Hệ thống Liên lạc và Xử lý Cuộc gọi}

\begin{columns}[t]
\begin{column}{0.5\textwidth}
\begin{block}{Vai trò chính}
\begin{itemize}
\item Thiết lập và quản lý kênh giao tiếp thoại
\item Hỗ trợ thông báo, cảnh báo bằng giọng nói
\item Tích hợp với các module phần mềm khác
\end{itemize}
\end{block}

\begin{alertblock}{Cấu hình hệ thống}
\begin{itemize}
\item \textbf{Phiên bản:} Asterisk 22.5.1
\item \textbf{Hệ điều hành:} Linux Mint 21
\item \textbf{Vai trò:} Tổng đài VoIP nội bộ
\end{itemize}
\end{alertblock}
\end{column}

\begin{column}{0.45\textwidth}
\begin{exampleblock}{Tệp cấu hình chính}
\textbf{pjsip.conf:} Quản lý endpoint và xác thực

\textbf{extensions.conf:} Định nghĩa Dial Plan và kịch bản xử lý

\textbf{manager.conf:} Cấu hình AMI cho tương tác Python
\end{exampleblock}
\end{column}
\end{columns}

\vspace{0.3cm}
\begin{center}
\textit{\small Asterisk đóng vai trò kênh kết nối, tiếp nhận tín hiệu và chuyển giao xử lý cho máy chủ}
\end{center}

\end{frame}

% Slide 2: Cấu hình PJSIP
\begin{frame}[fragile]
\frametitle{Cấu hình PJSIP và Endpoint}

\begin{columns}[t]
\begin{column}{0.55\textwidth}
\begin{minted}[fontsize=\tiny, frame=single, breaklines]{ini}
[transport-udp]
type=transport
protocol=udp
bind=0.0.0.0

[6001]
type=endpoint
disallow=all
allow=ulaw
auth=auth6001
aors=6001
context=internal

[auth6001]
type=auth
auth_type=userpass
username=6001
password=1234
\end{minted}
\end{column}

\begin{column}{0.4\textwidth}
\begin{block}{Thành phần chính}
\textbf{Endpoint [6001]:}
\begin{itemize}
\item Thiết bị cuối nhận tin báo
\item Context: internal
\item Codec: ulaw
\end{itemize}

\textbf{Endpoint [server]:}
\begin{itemize}
\item Điều phối SIP messages
\item Kết nối với Python server
\end{itemize}
\end{block}
\end{column}
\end{columns}

\vspace{0.2cm}
\begin{alertblock}{Chức năng}
PJSIP đảm bảo chỉ thiết bị hợp lệ kết nối và định nghĩa thông tin xác thực
\end{alertblock}

\end{frame}

% Slide 3: Dial Plan
\begin{frame}[fragile]
\frametitle{Cấu hình Dial Plan}

\begin{columns}[t]
\begin{column}{0.6\textwidth}
\begin{minted}[fontsize=\tiny, frame=single, breaklines]{ini}
[internal]
exten => 6001,1,Answer()
same => n,Dial(PJSIP/6001,20)
same => n,Hangup()

exten => 6000,1,Dial(PJSIP/6001&PJSIP/6002&PJSIP/6003,20)
same => n,Hangup()

[messages]
exten => _X.,1,NoOp(===> SIP MESSAGE from ${MESSAGE(from)})
same => n,MessageSend(pjsip:${EXTEN},pjsip:server)
same => n,NoOp(===> Send status: ${MESSAGE_SEND_STATUS})
same => n,Hangup()
\end{minted}
\end{column}

\begin{column}{0.35\textwidth}
\begin{block}{Context [internal]}
\begin{itemize}
\item Xử lý cuộc gọi nội bộ
\item Extension 6001: thiết bị đơn
\item Extension 6000: gọi nhóm
\end{itemize}
\end{block}

\begin{exampleblock}{Context [messages]}
\begin{itemize}
\item Xử lý tin nhắn SIP
\item Chuyển tiếp đến server
\item Kích hoạt xử lý Python
\end{itemize}
\end{exampleblock}
\end{column}
\end{columns}

\end{frame}

% Slide 4: AMI Configuration
\begin{frame}[fragile]
\frametitle{Cấu hình Asterisk Manager Interface (AMI)}

\begin{columns}[t]
\begin{column}{0.5\textwidth}
\begin{minted}[fontsize=\small, frame=single, breaklines]{ini}
[general]
enabled = yes
port = 5038
bindaddr = 127.0.0.1

[hx]
secret = 123
read = all
write = all
\end{minted}
\end{column}

\begin{column}{0.45\textwidth}
\begin{block}{Chức năng AMI}
\begin{itemize}
\item Giao diện lập trình cho ứng dụng ngoài
\item Cho phép Python điều khiển Asterisk
\item Lấy thông tin trạng thái cuộc gọi
\item Gửi lệnh từ máy chủ
\end{itemize}
\end{block}

\begin{alertblock}{Bảo mật}
\textbf{bindaddr = 127.0.0.1}
\begin{itemize}
\item Chỉ cho phép kết nối cục bộ
\item Ngăn chặn truy cập trái phép
\item Tài khoản riêng biệt [hx]
\end{itemize}
\end{alertblock}
\end{column}
\end{columns}

\end{frame}
