% ----------------------------
% Slide: Asterisk – Hệ thống Liên lạc SIP
% ----------------------------
\begin{frame}
\frametitle{Asterisk: Liên lạc và Xử lý Cuộc gọi SIP}
\scriptsize
\begin{tabular}{|p{2.3cm}|p{4cm}|p{3cm}|}
\hline
\textbf{Thông tin} & \textbf{Chi tiết} & \textbf{Chức năng} \\
\hline
Phiên bản & Asterisk 22.5.1 trên Linux Mint 21 & Tổng đài VoIP \\
\hline
pjsip.conf & Quản lý endpoint & Kết nối qua Python (AMI/ARI) \\
\hline
extensions.conf & Dial Plan & Xử lý cuộc gọi SIP \\
\hline
manager.conf & Cấu hình AMI & Python điều khiển cuộc gọi \\
\hline
\end{tabular}

\end{frame}


% Slide 2: Cấu hình PJSIP
\begin{frame}[fragile]
\frametitle{Asterisk: Cấu hình PJSIP và Endpoint}

\begin{columns}[t]
\begin{column}{0.55\textwidth}
\begin{minted}[fontsize=\tiny, frame=single, breaklines]{ini}
[transport-udp]
type=transport
protocol=udp
bind=0.0.0.0

[6001]
type=endpoint
disallow=all
allow=ulaw
auth=auth6001
aors=6001
context=internal

[auth6001]
type=auth
auth_type=userpass
username=6001
password=1234
\end{minted}
\end{column}

\begin{column}{0.4\textwidth}
\begin{block}{Thành phần chính}
\textbf{Endpoint [6001]:}
\begin{itemize}
\item Thiết bị cuối nhận tin báo
\item Context: internal
\item Codec: ulaw
\end{itemize}

\textbf{Endpoint [server]:}
\begin{itemize}
\item Điều phối SIP messages
\item Kết nối với Python server
\end{itemize}
\end{block}
\end{column}
\end{columns}

\vspace{0.2cm}
\begin{alertblock}{Chức năng}
PJSIP đảm bảo chỉ thiết bị hợp lệ kết nối và định nghĩa thông tin xác thực
\end{alertblock}

\end{frame}

% Slide 3: Dial Plan
\begin{frame}[fragile]
\frametitle{Asterisk: Cấu hình Dial Plan}

\begin{columns}[t]
\begin{column}{0.6\textwidth}
\begin{minted}[fontsize=\tiny, frame=single, breaklines]{ini}
[internal]
exten => 6001,1,Answer()
same => n,Dial(PJSIP/6001,20)
same => n,Hangup()

exten => 6000,1,Dial(PJSIP/6001&PJSIP/6002&PJSIP/6003,20)
same => n,Hangup()

[messages]
exten => _X.,1,NoOp(===> SIP MESSAGE from ${MESSAGE(from)})
same => n,MessageSend(pjsip:${EXTEN},pjsip:server)
same => n,NoOp(===> Send status: ${MESSAGE_SEND_STATUS})
same => n,Hangup()
\end{minted}
\end{column}

\begin{column}{0.35\textwidth}
\begin{block}{Context [internal]}
\begin{itemize}
\item Xử lý cuộc gọi nội bộ
\item Extension 6001: thiết bị đơn
\item Extension 6000: gọi nhóm
\end{itemize}
\end{block}

\begin{exampleblock}{Context [messages]}
\begin{itemize}
\item Xử lý tin nhắn SIP
\item Chuyển tiếp đến server
\item Kích hoạt xử lý Python
\end{itemize}
\end{exampleblock}
\end{column}
\end{columns}

\end{frame}

% Slide 4: AMI Configuration
\begin{frame}[fragile]
\frametitle{Asterisk: Cấu hình Asterisk Manager Interface (AMI)}

\begin{columns}[t]
\begin{column}{0.4\textwidth}
\begin{minted}[fontsize=\tiny, frame=single, breaklines]{ini}
[general]
enabled = yes
port = 5038
bindaddr = 127.0.0.1

[hx]
secret = 123
read = all
write = all
\end{minted}

\vspace{0.2cm}
\begin{alertblock}{}
\centering
\footnotesize
\textbf{Bảo mật: Chỉ localhost}\\
\texttt{bindaddr = 127.0.0.1}
\end{alertblock}
\end{column}

\begin{column}{0.55\textwidth}
\begin{table}[htbp]
\centering
\scriptsize
\begin{tabular}{|l|l|}
\hline
\textbf{Tham số} & \textbf{Chức năng} \\
\hline
\texttt{port 5038} & Cổng kết nối AMI \\
\hline
\texttt{[hx] account} & Tài khoản Python server \\
\hline
\texttt{read/write all} & Quyền điều khiển đầy đủ \\
\hline
\end{tabular}
\end{table}

\vspace{0.3cm}
\begin{exampleblock}{Vai trò AMI}
\begin{itemize}
\item Giao diện lập trình API
\item Python điều khiển Asterisk  
\item Lấy trạng thái cuộc gọi
\item Gửi lệnh quản lý
\end{itemize}
\end{exampleblock}
\end{column}
\end{columns}

\end{frame}
